\documentclass[12pt,oneside]{article}
\usepackage[top=1in, bottom=1in, left=1in, right=1in]{geometry} 
\usepackage{amsmath, amsthm}               		
\geometry{letterpaper}                   		
\usepackage{graphicx}				
\usepackage{amssymb}
\usepackage{subfig}
\usepackage{natbib}
\usepackage[nodisplayskipstretch]{setspace}

\newcommand{\fdir}{\string~/Dropbox/Washington_State_Field_Trip/Daily_Report_Tex}


\begin{document}

% Day 1 - August 3, 2014
\clearpage
\section{Day 1 -- August 3, 3014: Ophiolite complex on Fidalgo Island}

\textbf{10:10 am:} 	Arrived to the Seattle-Tacoma International Airport, picked up rental cars
\\\textbf{11:15 am:}	Left the airport, drove north on I-5
\\\textbf{11:45 am:} 	Stopped at Safeway for lunch and to buy food for the following 4 days 
\\\textbf{1:00 pm:}	Continued driving north on I-5 toward Mt. Vernon, then turned west on Route 20 to Anacortes (Fidalgo Island:)
\\\textbf{3:00 pm:}	Stop 1 -- the viewpoint at Cap Sante Park in Anacortes 
\\\textbf{3:40 pm:}	Continued driving west on Fidalgo Island to Washington Park
\\\textbf{3:55 pm:}	Stop 2 -- Green point, Washington Park
\\\textbf{4:25 pm:}	Left Green point and drove to Mt. Erie
\\\textbf{4:45 pm:}	Stop 3 -- Mount Erie viewpoint, Mount Erie Park 
\\\textbf{5:30 pm:}	Left Mt. Erie and drove south to Deception Pass State Park campground

\subsection{Stop 1 -- Cap Sante Park viewpoint in Anacortes}
This stop offers spectacular 360$^{\circ}$ views of Fidalgo Bay, the City of Anacortes, marinas, and the Cascades with snow-capped Mount Baker (Figure~\ref{Fig:080314_1}). The viewpoint is located on a hill made up by coarse-grained sedimentary rocks about 60 million years old. Although not a part of the Fidalgo ophiolite per se, these rocks represent near-shore sediments that could be found overlaying the oceanic crust. The outcrop allows examining various sedimentary fabrics and glacial features such as striations and glacially smoothed surfaces. One can also observe glacial till deposits down by the water to the northwest of the viewpoint, which stand out due to their lighter color and noticeably poor consolidation. Another interesting observation that can be made at this stop (as well as en route throughout the day) is the differences in vegetation on various islands scattered in the bay, which is related to bedrock geology. Ultramafic rocks, in particular, the low-crustal and mantle sections of the ophioliote complex, weather into soils poor in potassium and other essential nutrients, which cannot support vegetation. Therefore, the ophiolite islands can be recognized from the distance by the lack of vegetation on them. 

\begin{figure}
\centering
\makebox[\textwidth]{
\subfloat[]{\includegraphics[width=0.5\textwidth]{\fdir/080314_1a.png}}
\subfloat[]{\includegraphics[width=0.5\textwidth]{\fdir/080314_1b.png}}}
\caption{A view on Mount Baker (a) and an outcrop of meta-sedimentary rocks (b) at Cap Sante Park's viewpoint.}
\label{Fig:080314_1}
\end{figure}

\subsection{Stop 2 -- Green point, Washington Park}
The name of this stop comes from the exposures of serpentinite, a metamorphosed mantle rock, found along a picturesque rocky shore in Washington park. It represents an upper-mantle section in the basal portion of the Fidalgo ophiolite. The rocks that we examined on the northern side of the beach at Green point had rusty-brown color characteristic of weathered mantle rocks in many ophiolite outcrops. We observed dunites and pyroxenites intersected by extensive network of carbonate veins (Figure~\ref{Fig:080314_2}). The ophiolite rocks at this location are overlain by loose glacial deposits.

\begin{figure}
\centering
\makebox[\textwidth]{
\subfloat[]{\includegraphics[width=0.5\textwidth]{\fdir/080314_2a.png}}
\subfloat[]{\includegraphics[width=0.5\textwidth]{\fdir/080314_2b.png}}}
\caption{Mantle section of Fidalgo ophiolite at Green point, Washington Park: pyroxenite banding in dunite (a) and carbonate veins (b).}
\label{Fig:080314_2}
\end{figure}

\subsection{Stop 3 -- Mount Erie viewpoint, Mount Erie Park}
The rocks found on Mt. Erie represent the crustal section of the Fidalgo ophiolite that lie sequentially above the mantle serpentinite at stop 2. Here they consist mainly of diorite, a slightly more felsic rock than gabbro commonly found in ophiolite�s crustal section. They are interpreted to represent magmas formed below a chain of subduction-zone volcanoes. U-Pb zircon ages are 167 � 5 Ma ~\citep{Tucker2014}. Thus, the Fidalgo ophiolite is interpreted to represent the lithosphere of a Jurassic volcanic arc. The whole sequence was tipped on its side during accretion, so the deepest portion, the serpentinite, is now found to the west (Washington Park), the uppermost portion, i.e. the ocean floor lavas and sedimentary rocks, are found toward the east, and the intracrustal intrusives are in the middle (Mt. Erie). The stop also offers spectacular views from small platforms on the summit knobs (Figure~\ref{Fig:080314_3}). 

\begin{figure}
\centering
\makebox[\textwidth]{
\subfloat[]{\includegraphics[width=1.0\textwidth]{\fdir/080314_3.png}}}
\caption{A view south from stop 3 at Mount Erie on Lake Campbell and Skagit Bay.}
\label{Fig:080314_3}
\end{figure}

% Day 2 - August 4, 2014
\clearpage
\section{Day 2 -- August 4, 2014}

\textbf{9:00 am:}		Review before leaving campsite
\\\textbf{10:15 am:}	Stop 1 -- Bedrock quarry near Sharpe Road
\\\textbf{11:04 am:}	Stop 2 -- Bridges at Deception Pass
\\\textbf{1:00 pm:}	Stop 3 -- Rosario Beach, Deception Pass State Park
\\\textbf{4:30 pm:}	Stop 4 -- Location of Oso landslide
\\\textbf{4:45 pm:}	Stop 5 -- Stop on side of the road, 6 mi W of Darrington
\\
\\
In preparation for what we will see today, we reviewed the `classic' structure of oceanic crust that is expected within an ophiolite sequence, although in practice it is rarely all observed (Figure~\ref{fig:ophiolite}). We re-familiarised ourselves with the classification of ultramafic rocks, several of which we will see today (Figure~\ref{fig:ultramafic}). We re-familiarised ourselves with the classification of plutonic igneous rocks, several of which we will see over the course of the field trip (Figure~\ref{fig:cumulate}). Note the important evolutionary sequence from gabbro/diorite, through tonalite, to granodiorite, and finally granite.

\begin{figure}[h]
\centering
\includegraphics[trim=2.8cm 5.5cm 2.5cm 3.2cm,clip=true,width=0.8\textwidth]{\fdir/ophiolite_edited}
\label{fig:ophiolite}
\caption{Ophiolite Sequence}
\end{figure}

\begin{figure}[h]
\centering
\includegraphics[trim=0cm 1.2cm 0cm 0.4cm,clip=true,width=0.55\textwidth]{ultramafic.jpg}
\label{fig:ultramafic}
\caption{Classification of ultramafic rocks. Image credit: J D Price, 2011}
\end{figure}

\begin{figure}[h]
\centering
\includegraphics[width=0.55\textwidth]{plutonic.jpg}
\label{fig:cumulate}
\caption{Classification of plutonic igneous rocks}
\end{figure}

\subsection{Stop 1 -- Bedrock quarry near Sharpe Road}

\subsubsection{10:15 am -- Outcrop in the heather on the side of the hill just below the concrete foundation.}
\underline{Initial observations:}\\
Rocks are hard and crystalline, with sugary texture. A dark-green matrix includes dark, 1-3mm crystals exhibiting 90$^{\circ}$ cleavage: these are pyroxenes. \\
\underline{Explanation:}\\
The colour and the texture indicate that this outcrop consists of a de-vitrified lava; the green colours may be attributed to the minerals chlorite and epidote, which grew during metamorphism. The larger pyroxenes result from recrystallisation during metamorphism.\\
\underline{Summary:}\\
This outcrop comprises meta-volcanics from uppermost extrusive sequence beneath the pillow lavas (not seen here).\\
\underline{Other notes:}\\
Some trace veins also evident. Whitish colour, hard to scratch - probably quartz. 

\subsubsection{10:35 am -- 10m up section from last stop, above the concrete foundation.}
\underline{Initial observations:}\\
Rocks are less hard and generally lighter. They contain small clasts, which are rounded and comprise non-interlocking crystals. Some quartz crystals are observed. The weathering of this outcrop is somewhat distinct from the previous stop, with more bulging rounded faces.\\
\underline{Explanation:}\\
The clastic nature of this rock indicates that it had a sedimentary protolith, but later metamorphism is evident from texture and the presence of some alteration minerals.\\
\underline{Summary:}\\
This outcrop comprises slightly metamorphosed volcanoclastic sediments at the top of an obducted oceanic crustal section.\\
\underline{Other notes:}\\
Outcrop is heavily weathered and overgrown with lichen\footnote{Which, as we'll learn later, is a symbiotic amalgam of a fungus and an algae.}. Hard to obtain a fresh face.

\subsubsection{10:50 am -- Just next to parking spot on the way back from the last two stops.}
\underline{Initial observations:}\\
Very fine-grained, dark rock, which cleaves easily. Foliation evident in places, as well as suggestion of initial bedding.\\
\underline{Explanation:}\\
The absence of large clasts and homogeneity of fine grain size, as well as the colour, indicates that this rock is a mudstone/argillite. The strong foliation along which it cleaves is a product of metamorphism to a slate texture. \\
\underline{Summary:}\\
This outcrop comprises metamorphosed pelagic ocean sediments with unclear stratigraphic relationship to the volcanoclastics we saw last stop.\\
\underline{Other notes:}\\
The colour of this mudstone is mostly dark, but much more reddish in some layers, presumably due to Iron content.

\subsection{Stop 2 -- Bridges near Deception Pass}

\subsubsection{11:05 am -- North side of bridges at Deception Pass. Phenomenal exposure in cliffs above road.}
\underline{Initial observations:}\\
Grey/green altered sedimentary rock with clear bedding features and quartz veins that are folded and boudinaged. Small faults cross-cut the cliff in multiple directions, with several good examples of slickensides particularly evident in the quartz veins (Figure~\ref{Fig:day2_stop4_fig1}). The bedding dips steeply ($\sim40^{\circ}$) into the cliff face (East). Further up (N) the bedding pattern is clearer, with alternating sequences of pebbly conglomerates grading into (green/greyish) sandstones and thence to (darker) mudstones.\\
\underline{Explanation:}
The alternating fining-up sequences comprise a series of turbidites. There was some disagreement about the younging direction, which should in theory be in the direction of the fining. It is possible that the heavily faulted and folded nature of this outcrop resulted in different younging directions in different places. This outcrop included several fault zones with a thrusting sense (see field sketch), and the major fault structures dipped northwards, implying N-S convergence. Pervasive deformation is also indicated by the strongly folded bedding planes and boudinage of more competent vein material.  \\
\underline{Summary:}\\
This outcrop is a heavily deformed turbidite sequence, with some well-preserved brittle and ductile deformation structures.\\
\underline{Other notes:}\\
Excellent spot for lunch on Pass Island w/ remarkable views. Little shade from the sun. 

\begin{figure}
\centering
\makebox[\textwidth]{
\subfloat[Outcrop on NE side of road, N of bridges]{\includegraphics[width=0.5\textwidth]{day2_stop4_map}}
\subfloat[Field sketch of small fault (total height 40 cm) with displaced quartz veins.]{\includegraphics[width=0.5\textwidth]{day2_stop4_fig1}}}
\caption{}
\label{Fig:day2_stop4_fig1}
\end{figure}

\subsection{Stop 3 -- Rosario Beach, Deception Pass State Park}

\subsubsection{1:00 pm -- North of Rosario Beach, on rocks along the coast}
\underline{Initial observations:}\\
Clearly bedded rock with black/dark layers (2-10 cm thick) of fine-grained material alternating with harder, sugary, dark layers. Lots of small folds are evident, including isoclinal folds, as is boudinage of the harder material (Figure~\ref{fig:day2_stop5_fig1}). \\
\noindent \underline{Explanation:}\\
This rock comprises mud/silt stone layers, that are soft and weather preferentially, interbedded with harder cherty layers that weather proud. The deposition of cherts indicates seafloor deposition (of diatoms/radiolaria) on seafloor deeper than the CCD. The intense deformation into beautiful structures indicates that that this rock has been highly strained.\\
\underline{Summary:}\\
This outcrop is layered cherts and mudstones that have been heavily folded and lightly metamorphosed.\\
\underline{Other notes:}\\
Fun and easy climbing and lovely small-scale folding structures.

\begin{figure}[h]
\centering
\includegraphics[width=0.55\textwidth]{day2_stop5_photo1}
\label{fig:day2_stop5_fig1}
\caption{Banded cherts showing small-scale folding.}
\end{figure}

\subsubsection{1:30 pm -- 30 m above last point, on top of the headland.}
\noindent \underline{Initial observations:}\\
Bumpy, black, fine-grained crystalline rock outcropping in scrubby grass. The ``bumps'' are between 30 and 60 cm in length. Some are broken open and appear to have radial fracture lines, with zoning of colour. Less than 20 m all around this outcrop we find more of the interbedded cherts/mudstones from the previous stop.\\
\underline{Explanation:}\\
These are pillow basalts! They are fine grained because of the rapid cooling, and dark coloured because of their mafic composition. The colour zonation is likely due to differential weathering at different radii from the centre of the pillows that results from different cooling times and hence initial vesicularly etc.. The radial fractures are a result of tensional stresses introduced by rapid quenching with radial heat flow. Observations of the cliffs on the west face of the headland show that these pillow basalts cross-cut the sedimentary layers from the previous stop, indicating that the pillows have intruded (Figure~\ref{fig:day2_stop6_fig1}).\\
\underline{Summary:}\\
Pillow basalts from the top of the ophiolite sequence, intruded into the surrounding abyssal sediments.\\
\noindent \underline{Other notes:}\\
Excellent view of the Bull Kelp bloom (Figure~\ref{fig:day2_stop6_fig2}). These kelp die each year and re-grow during the summer at rates of 10'' per day! They are rooted to the bottom and provide an important habitat for wildlife, including manatees, and breeding fish. The most recent day's growth has a different appearance and is edible. 

\begin{figure}[h]
\centering
\includegraphics[width=0.55\textwidth]{day2_stop6_photo1}
\label{fig:day2_stop6_fig1}
\caption{Side of cliff, showing pillow basalts cross cutting the cherty layers.}
\end{figure}

\begin{figure}[h]
\centering
\includegraphics[width=0.55\textwidth]{day2_stop6_photo2}
\label{fig:day2_stop6_fig2}
\caption{Kelp beds in channel below cliff.}
\end{figure}

\subsection{Stop 4 -- Oso landslide}
\underline{Initial observations:}\\
We drove past (one cannot stop) the landslide deposit from the 22nd March 2014 land/mud-slide. Currently the whole area is covered with construction crews, but once can see the incredibly long runout, where all the trees and the town were totally demolished.\\
\underline{Explanation:}\\
Forty-three people were killed as the SW flank of an unstable hill collapsed, resulting in a mud and debris flow running out for more than half a mile, covering the rural neighbourhood of Steelhead Haven. The suspected cause was soil saturation and weakening following heavy rainfall. There is a distinctive hummocky terrain to the runout deposit\footnote{Which we would later see echoed at volcanic landslides.} due to coherent blocks within the flow.\\
\underline{Summary:}\\
Destructive and humbling runout deposit of a recent, historic landslide (Figure~\ref{fig:day2_stop7_photo1}).\\
\underline{Other notes:}\\
The diminutive hill from which the landslide came belies the destruction wrought.

\begin{figure}[h]
\centering
\includegraphics[width=0.8\textwidth]{day2_stop7_photo1}
\label{fig:day2_stop7_photo1}
\caption{Oso landslide provenance and deposit from car.}
\end{figure}

\subsection{Stop 5 -- Stop on side of the road, 6 mi W of Darrington}
\underline{Initial observations:}\\
Looking up at the mountains to our north, we observed large scale structures a the crest of the ridge, and evident within the trees on the south slope. A stair-step morphology to the ridge crest indicated large-scale layering, dipping to the west. A change in the dip of the bedding was evident with apparently steeper dips in the ridge further west.\\
\underline{Explanation:}\\
This large scale layering is found within a Mesozoic sedimentary section that comprises the accretionary wedge. The change in dip of the layers could be a result of a SW striking, NW-dipping thrust fault - see field sketch (Figure~\ref{fig:day2_stop8_fig1}). Major SW-striking faults are expected in this region, related to continental margin convergence, but should be normal in sense. Either our ``Swiss geology'' from afar is inaccurate, or this might be a reactivated structure.  \\
\underline{Summary:}\\
Mesozoic accretionary sequence seen from miles away, on ridge line.\\

\begin{figure}[h]
\centering
\includegraphics[width=0.8\textwidth]{day2_stop8_fig1}
\label{fig:day2_stop8_fig1}
\caption{Field sketch of ridge}
\end{figure}

% Day 3 - August 5, 2014
\clearpage
\section{Day 3 -- August 5, 2014: Mt. Baker from afar}

\textbf{9:40 am:	}	Leave campground
\\\textbf{10:30 am:}	Stop 1 -- Mt. Baker Overlook on the way to Anderson-Watson Trail
\\\textbf{11:45 am:}	Stop 2 -- Anderson-Watson Trail
\\\textbf{3:20 pm:}	Stop 3 -- Upper Baker Lake Dam
\\\textbf{4:20 pm:}	Stop 4 -- Boulder Creek 
\\\textbf{5:00 pm:}	Stop 5 -- Hot Springs
\\
\\
After a quick breakfast, we leave the campground, driving south on some road.  On the right hand side, are lavas of the Sulfur Creek stage which are mostly covered by trees.  We drive by the Upper Baker Lake Dam, a hydroelectric plant.  The dam walls expose layers of metasedimentary rocks.  As we drive up the mountain, a brave little quail runs in front of the car, bringing the caravan to a halt.  Jonathan has to run out of the car to encourage the quail to leave the road.

\subsection{Stop 1 -- Mt. Baker Overlook, 10:30--11:30 am}
We stop on a mountain road to talk a little bit about the geology and glaciology of the Mt. Baker area and discuss how Mt. Baker (to the NW, Figure~\ref{Fig:080514_1}) fits into the context of the Cascade Range.  The lithology present at the overlook is primarily phyllite, which can be seen from the micaceous sheen as well as the cleavage. 

\begin{figure}[h]
\centering
\includegraphics[width=0.8\textwidth]{080514_1.png}
\label{Fig:080514_1}
\caption{Mt. Baker from the overlook.}
\end{figure}

\noindent Looking down into the valley, we see Anderson Creek.  The bedrock here is primarily Cretaceous Shuksan Greenschist, which continues to Mt. Shuksan to the North.  The Shuksan Greenschist is bounded to the East by the Stray Creek dextral strike-slip fault.  This fault is one of a series of faults that are responsible for smearing old terrains in the Western United States hundreds of kilometers northward as a result of partitioning due to oblique subduction in this region (see the Geologic History of the Pacific Northwest section of this guide).  Underlying Mt. Shuksan is the older Jurassic Nooksack Formation composed of sandstones and ardulites.  \\\\
On Mt. Shuksan, we see two large glaciers (Sulfide Glacier and Crystal Glacier).  White Salmon Glacier is further to the West.  From our viewpoint, we clearly see Mt. Baker�s Park Glacier across the valley, which exhibits a clearly visible ablation zone where we see blue ice exposed where the overlying snowpack had melted.  This blue ice is the result of the metamorphism of ice at the bottom of the glacier.  Because the blue ice is darker than the white snow, it absorbs more radiation than the overlying snow.  This (as well as the even darker rock fall deposits) account for some of the variability in melting in different parts of the glaciers.   \\\\
Now, in preparation for our upcoming visits to Mts. Ranier and St. Helens, we paused for a quick overview of the Cascade Range.  The northern extent of the Cascade Range is composed of the Garibaldi Volcanic Belt, which consists of three volcanoes in Canada and Mt. Baker and Glacier Peak in the United States.  The northern volcanoes have less output than the more southern components of the chain and Mt. Baker is the first volcano going south that appears to have significant output.  Volcanoes further south in the Cascade Range generally have larger output (or better preservation of erupted material?) than those further to the north.  In the Mt. Baker area, this effect is reflected by the paucity of cindercones that contribute to volcanic output.  In addition to the marked change in output represented by Mt. Baker, the volcano is also unique in that it is the most mafic of the volcanoes in the Garibaldi Volcanic Belt suggesting a more open pathway which allows magma to travel to the surface more 	quickly.\\\\
Mt. Baker�s edifice is about 20 ka and has been growing for the last 40,000 years.  The most recent eruption at Mt. Baker was ~5,900 years ago.  Some crags on the volcano (the Black Buttes) are slightly offset to the SW suggesting older episodes of volcanism and have been dated to ~0.5 Ma.  Further to the east is evidence for more destructive caldera-style episodes (~1 Ma).  This is marked by ignimbrites flowing into the caldera and filling in the Kulshan Caldera depression created by the eruption.      

\subsection{Stop 2 -- Anderson-Watson Trail, 11:45 am -- 3:00 pm}

We went for a hike further up the road from Stop 1.  The Anderson-Watson Trail is highly recommended for anyone who is interested in a ~ 2 hour roundtrip hike (albeit with  $\sim \pm 500$ ft elevation change) culminating in an idyllic view of Mt. Baker.  Along the trail, we are various volcanic and metasedimentary rocks including some metamorphosed rocks with nice augen, which offer a nice opportunity to practice interpreting strain markers.  We stopped for lunch in an alpine meadow surrounded by flowers with a beautiful view of Mt. Baker (Figure~\ref{080514_2}).

\begin{figure}[h]
\centering
\includegraphics[width=0.8\textwidth]{080514_2.png}
\label{Fig:080514_2}
\caption{Lunch stop on the Anderson-Watson trail.}
\end{figure}

\subsection{Stop 3 -- Upper Baker Lake Dam, 3:20 pm}
Here, we saw metamorphosed mudstone (slate or phyllite) of the Nooksack Formation with EW striking quartz veins and faults (180/57W) with foliation dipping to the north (253/56N).  The outcrop exhibits good foliation and early mica development.  Walking over the bridge over the dam, we discussed hazard preparedness in association with volcanic activity at Mt. Baker.  In 1975, Mt. Baker started emitting elevated levels of SO2.  In response to this increased activity, officials began improving volcanic monitoring instrumentation in the area (for example installing a seismic network to monitor volcanic tremor) and planning for hazard response in the event of an eruption.  For example, they realized that, in anticipation of an eruption, dam levels should be lowered to reduce the risk from lahars and pyroclastic flows, which would follow river channels away from the volcano.  Though the 1975 Mt. Baker eruption never happened, the increased awareness that followed in the wake of the volcanic activity proved to be important 5 years later when Mt. St. Helens erupted in 1980.  Richard, a hot springs soaker from Hamilton, confirmed that flooding hazards associated with the dam are a constant concern and that major flooding affects locals every few years.    

\subsection{Stop 4 -- Boulder Creek, 4:20 pm}
A small trail to the side of the bridge here leads to a streambed.  The boulders making up the streambed are primarily black and red in color.  Both are basaltic in composition, the red color deriving from the in situ oxidation of the volcanic debris (Figure~\ref{Fig:080514_4}).  Walking along this streambed, we reached an outcrop of lahar deposits from an 1845 cold lahar (a lahar not associated with a volcanic eruption).  These deposits are poorly sorted.  One very notable feature of these lahar deposits is a large tree trunk lying within the deposit (Figure~\ref{Fig:080514_4}).

\begin{figure}[h]
\centering
\makebox[\textwidth]{
\subfloat[Oxidized rocks at Boulder Creek]{\includegraphics[width=0.5\textwidth]{080514_4a.png}}
\subfloat[Tree trunk preserved in a cold lahar deposit at Boulder Creek.]{\includegraphics[width=0.5\textwidth]{080514_4b.png}}}
\label{Fig:080514_4}
\caption{}
\end{figure}

% Day 4 - August 6, 2014
\clearpage
\section{Day 4 -- August 6, 2014: North Cascades National Park}
\textbf{8:30 am:	}	Discussion at the car before leaving campsite
\\\textbf{8:45 am:	}	Depart Horseshoe Cove Campground
\\\textbf{10:15 am:}		Stop 1 -- New Halem
\\\textbf{10:40 am:}		Stop 2 -- Gorge Creek Outlook
\\\textbf{11:15 am:}		Stop 3 -- Outcrop at Mile 124.5 on Route 20
\\\textbf{11:30 am:}		Stop 4 -- Outcrop at Mile 125.2 on Route 20
\\\textbf{12:00 pm:}		Stop 5 -- Diablo Lake Overlook, lunch
\\\textbf{1:00 pm:}	Stop 6 -- John Pierce Falls
\\\textbf{1:45 pm:}	Stop 7 -- View over Ross Lake
\\\textbf{2:30 pm:}	Stop 8 -- Washington Pass Overlook
\\\textbf{3:15 pm:}	Stop 9 -- Outcrop at Mile 167.6 on Route 20
\\\textbf{5:40 pm:}	Arrive at Daroga State Park
\\\\
Leaving Horseshoe Cove Campground, we drive south on Baker Lake Road and then southeast on Burpee Hill Road towards Concrete where we turn east onto Route 20 (North Cascades Highway). We drive through the Sulphur Creek Lava Flow, a subaerial lava flow from Mt. Baker erupted approximately 8000-9000 years ago. We also drive through parts of the Nooksack Formation, Jurassic sedimentary rocks that form the basement of Mt. Baker. Driving up Burpee Hill, we pass through pyroclastic flow deposits from Mt. Baker.\\\\
Between Concrete and Marblemount, we see landslides and fluvial deposits on the north side of the road. At Marblemount, we make a sharp left turn on Route 20 to follow the Straight Creek Fault. This fault separates Mesozoic Metasediments to the west and the 220 Ma Marblemount Meta-Quartz Diorite to the east. The Quartz Diorite is a muscovite-chlorite-epidote-alibet-quartz rock that contains small amounts of biotite and hornblende. This and equivalent plutonic formations extend southeastward from the Straight Creek Fault for ~150 km before disappearing beneath Miocene Columbia River Basalts. After entering North Cascades National Park, there are outcrops of Napeequa Schist around miles 112-113 on Route 20. Between miles 115-116 are outcrops of Oligocene granodiorite, exhibiting a rounded weathering pattern. We continue driving east into the Skagit Gneiss Complex, which is ~60-70 Ma. The first 7 stops of the day are all within this complex.

\subsection{Stop 1 -- New Halen, 10:15 am}
The outcrop is on the left when you enter New Halem, just after the visitor�s center. The New Halem orthogneiss is a $\sim$75-60 Ma biotite gness, containing a complicated network of quartz veins and fractures (Figure~\ref{Fig:stop1a}. The wavy quartz veins are migmatite, indicating that the rock started to melt on a large scale. The straighter pegmatite dikes are evidence of a later hydrothermal overprint. On a smaller scale, biotite is drawn out into concentrated layers. This gneiss is in the lower to middle part of the crustal section.

\begin{figure}[h]
\centering
\includegraphics[width=0.8\textwidth]{STOP1A.jpg}
\label{Fig:stop1a}
\caption{Migmatite quartz veins in the New Halem orthogneiss.}
\end{figure}

\subsection{Stop 2 -- Gorge Creek Outlook, 10:40 am}
The outcrops are a short walk from the parking lot. The Skagit orthogneiss is more strongly foliated than the previous outcrop, indicating more deformation relative to the New Halen orthogneiss. There is little migmatization, but evidence is present of both brittle and ductile deformation. There are large fault surfaces with slickensides (Figure~\ref{Fig:stop2a}) as well as smaller-scale folds. Large radial blast fractures in the outcrop have an obvious anthropogenic source. After examining the outcrops at the overlook, cross the road and walk out onto the bridge. There is a nice view of a tall, narrow waterfall.

\begin{figure}[h]
\centering
\includegraphics[width=0.5\textwidth]{STOP2A.jpg}
\label{Fig:stop2a}
\caption{Slickensides on a fault surface.}
\end{figure}

\subsection{Stop 3 -- Outcrop at Mile 124.5 on Route 20, 11:15 am}
This outcrop is still within the Skagit gneiss complex. It is strongly banded, with mica dominating the darker layers. The lighter layers contains large clasts of augen. There are sigma shear markers indicating dextral shear (Figure~\ref{Fig:stop3}). There are many small garnets within the dark layers. This is a porphyroclastic augen gneiss that has undergone multiple phases of shear deformation.

\begin{figure}[h]
\centering
\makebox[\textwidth]{
\subfloat[]{\includegraphics[width=0.5\textwidth]{STOP3A.jpg}}
\subfloat[]{\includegraphics[width=0.5\textwidth]{STOP3B.jpg}}}
\label{Fig:stop3}
\caption{Shear markers in augen gneiss}
\end{figure}

\subsection{Stop 4 -- Oucrop at Mile 125.2 on Route 20, 11:30 am}
This gneiss outcrop contains significantly more garnets than the previous orthogneiss, indicating both higher-pressure conditions and a different, more aluminum-rich, source rock. This is a $\sim$70-68 Ma metapelite, with a shale parent rock (Figure~\ref{Fig:stop4a}).

\begin{figure}[h]
\centering
\includegraphics[width=0.8\textwidth]{STOP4A.jpg}
\label{Fig:stop4a}
\caption{Garnet-rich metapelite}
\end{figure}

\subsection{Stop 5 -- Diablo Lake Overlook, 12:00 pm}
First, we have lunch overlooking Diablo Lake until approximately 12:40. Then we walk to the outcrop directly across the road from the overlook. The outcrop contains many intersecting dikes indicating a large melt network (Figure~\ref{Fig:stop5}). These are quartzo-feldspathic pegmatites within the gneiss that date to 53 Ma. The gneiss itself is approximately 61 Ma, but there are older zircons within the gneiss that date to 2.6 Ga, demonstrating recycling of the crust. This gneiss is not strongly banded and is somewhat faulted.

\begin{figure}[h]
\centering
\makebox[\textwidth]{
\subfloat[View over Diablo Lake]{\includegraphics[width=0.5\textwidth]{STOP5A.jpg}}
\subfloat[Pegmatites in gneiss]{\includegraphics[width=0.5\textwidth]{STOP5B.jpg}}}
\label{Fig:stop5}
\caption{}
\end{figure}

\subsection{Stop 6 -- John Pierce Falls, 1:00 pm}
This gneiss outcrop is on the side of the road. It is very strongly banded and contains large amounts of mica. There are many small-scale folds. Pegmatite bodies dating to 54-52.5 Ma are pervasive, though they are not as large as at the previous stop. Lenses of leucosome, very light-colored rock, have formed within the bands. A large dike has been intruded into the gneiss; there is a clear contact on either side of the dike between the fine-grained dike and the coarser-grained gneiss. The dike itself is granitic to granodioritic and it is slightly foliated, particularly on the sides, because it was pushed through into the gneiss. On either side of the dikes, the density of pegmatites is high and decreases with distance from the intruded rock. There are also smaller dikes present (Figure~\ref{Fig:stop6}). This meta-greywacke outcrop is a good view of large-scale melt production. 

\begin{figure}[h]
\centering
\makebox[\textwidth]{
\subfloat[Small fold]{\includegraphics[width=0.5\textwidth]{STOP6A.jpg}}
\subfloat[Dike intruded into gneiss]{\includegraphics[width=0.5\textwidth]{STOP6B.jpg}}}
\label{Fig:stop6}
\caption{}
\end{figure}

\subsection{Stop 7 -- View over Ross Lake, 1:45 pm}
Last stop heading out of the Skagit Gneiss Complex, with a nice view over Ross Lake (Figure~\ref{Fig:stop7}). The Ross Lake Fault marks the end of the complex. The complex is bounded on the other side by the Entiat Fault. After this stop, we begin driving through the Golden Horn Batholith, which is a 49 Ma granitic pluton that is not genetically related to the Skagit Gneiss Complex. Note that as we head into the more granitic rocks, the weathering patterns change. With increasing amounts of k-feldspar, the color of the weathered rocks becomes more peachy-beige as the rocks weather to clay. The drainages begin to have a beige color that is typical of granite weathering.

\begin{figure}[h]
\centering
\makebox[\textwidth]{
\subfloat[View over Ross Lake]{\includegraphics[width=0.5\textwidth]{STOP7A.jpg}}
\subfloat[Cars parked at viewpoint]{\includegraphics[width=0.5\textwidth]{STOP7B.jpg}}}
\label{Fig:stop7}
\caption{}
\end{figure}

\subsection{Stop 8 -- Washington Pass Overlook, 2:30 pm}
Short walk from the parking lot to an overlook in the middle of the Golden Horn Batholith. We can see Liberty Bell to the right and the Early Winter Spires just to the left of Liberty Bell. Kangaroo Ridge is directly in front of us, behind a glacially carved valley (Figure~\ref{Fig:stop8a}). Decompression fractures are evidence for how the pluton has been unroofed (Figure~\ref{Fig:stop8bc}. We also see strange trees that grow in spirals in both directions, but primarily clockwise. One hypothesis for this is strong unidirectional wind leading the trees to grow that way, but we are not sure.\\
Note that this is also a nice spot for lunch, if you leave earlier in the morning or spend less time at the previous stops.

\begin{figure}[h]
\centering
\includegraphics[width=1.0\textwidth]{STOP8A.jpg}
\label{Fig:stop8a}
\caption{View from Washington Pass Overlook}
\end{figure}

\begin{figure}[h]
\centering
\makebox[\textwidth]{
\subfloat[Spiral trees]{\includegraphics[width=0.5\textwidth]{STOP8B.jpg}}
\linebreak
\subfloat[Decompression fractures]{\includegraphics[width=0.5\textwidth]{STOP8C.jpg}}}
\label{Fig:stop8bc}
\caption{}
\end{figure}

\subsection{Stop 9 -- Outcrop at Mile 167.6 on Route 20, 3:15 pm}
This outcrop by the side of the road is a light-colored granite, containing a large amount of k-feldspar. There is a rapakivi texture in the granite, with the pink feldspar crystals inside white plagioclase crystals (Figure~\ref{Fig:stop9a}. There are also some fine-grained mafic enclaves with amphiboles in plagioclase. At this stop, we also compare the granite to samples that Philipp collected at several outcrops along the road. We compare samples of tonalite, granodiorite, and granite. The fraction of k-feldspar is larger in the more granitic samples, while the granodiorite is dominated by quartz and the tonalite has only a very small fraction of k-feldspar. This indicates increasing fractionation of the melt as we move from tonalite to granite. The granite represents the shallow crust, while granodiorite and tonalite represent the shallow to middle level of the crust.

\begin{figure}[h]
\centering
\includegraphics[width=0.5\textwidth]{STOP9A.jpg}
\label{Fig:stop9a}
\caption{Rapakivi texture in granite}
\end{figure}


% Day 5 - August 7, 2014
\clearpage
\section{Day 5 -- August 7, 2014: }

% Day 6 - August 8, 2014
\clearpage
\section{Day 6 -- August 8, 2014: Hiking Burroughs Mountain at Mt. Rainier National Park}

\textbf{8:40:}		Depart campsite 
\\\textbf{9:05:} 	Stop 1 -- Ice contact lava columns. Road side between White River Campground and Sunrise. 
\\\textbf{10:30:}	  	Stop 2 -- Parking lot at Sunrise
\\\textbf{11:00:}		Starting hiking 
\\\textbf{11:30:} 		Stop 3 --  Little ice age moraine of Emmons glaciers 
\\\textbf{12:30:} 		Stop 4 --  On top of 2nd Burroghs Mtn.
\\\textbf{1:00:} 		Lunch 
\\\textbf{1:30:}		Stop 5 -- Pumice of the C tephra
\\\textbf{1:51:}		Stop 6 --  Igneous inclusion in andesite
\\\textbf{2:15:}		Stop 7 --  Back into granodiorite basement
\\\textbf{3:30:}		Stop 8 -- Old mine at Glacial Basin
\\\textbf{4:30:}		Back to White River Campground and drove back to campground
\\\textbf{5:30:} 		Return to campground
\\
\\
\noindent This day was dedicated to a 6 hour hiking on to the Burroughs Mountain at the northeastern side of Mt. Rainier National Park (Figure~\ref{Fig:080814_9}). Total hiking distance is about 9 mi. The trail starts with a elevation climb from 6400 ft at Sunrise to 7400 ft at top of 2nd Burroughs Mtn (3.2 mi), followed by a descend to 4232 ft at White River Campground(6 mi).  We were joined by USGS scientist Tom Sisson, who gave a very nice and knowledgeable lecture of volcanism at Mt. Rainier.  
\\
\\
8:40	am -- We left campsite, drove to White River Campground (the end of the trail, elevation 4232 ft). We parked one car here in order to shuttle drivers to the upper parking lot at Sunrise after the hike. 

\subsection{Stop 1 -- Roadside of Sunrise Park Rd between White River Campground and Sunrise, 9:30 am}
On the downhill side of the road we saw andesitic lava flows from the modern Mt. Rainier ($<$ 500,000 ybp). Tom pointed out that large volume lava flows are mostly found on ridge crests instead of valley floors. This is intriguing because lava flow tend to fill lower valley floor first. One explanation was that the valley-filling lava had been eroded away. But given the young age ($\sim$40,000 ybps) of these lavas, there is not enough time to erode away hundreds of meters of rocks. So this model was rejected.
\\\\
These ridge-forming lava flows have narrow, glassy columns pointing perpendicular to the ridge slope (Figure~\ref{Fig:080814_1}). These are actually cooling features resulting from lava being chilled against glacier ice. The direction of columns corresponds to direction of maximum cooling, indicating that lava was cooled from below.  In the valleys where ice was thick, lava flows were cooled rapidly, preventing it from flowing downhill. On the ridges where ice was thin, lava can melt away the ice and advance further. After the glacier retreated, the lava flows are left on the ridge.  The fact that these ice contact features are ubiquitous and one rarely see valley floor lava suggest that these valleys were covered with thick ice during nearly all of Pleistocene. The top of these lava flows can be used to infer ice thickness at the time of the eruption.

\begin{figure}[h]
\centering
\includegraphics[width=0.8\textwidth]{080814_1.png}
\label{Fig:080814_1}
\caption{Lava cooling feature at Stop 1}
\end{figure}

\noindent We also saw some plutonic inclusions within the andesitic lava. These are porous, and are compositionally more mafic than the lava (according to Tom). These inclusions have been dated to be around the same age of the erupted lava. This suggests that these are genetically related with the lava (although slightly older), possibly included during some magma mingling activity. 

\subsection{Stop 2 -- parking lot at Sunrise (elevation 6400 ft), 10:30 am}
As we are getting ready for the long hike, Tom gave us an intro of what we see at this side of Mt Rainier (Figure~\ref{Fig:080814_2}).  There are two summit craters at Rainier: East crater which is taller and younger (in the picture) and west crater. Both craters have steam vents with temperature measured to be up to 100 �C.  This is reportedly a subglacial crater lake inside the East crater. Below the summit we see Emmons glacier to the left (east) and Winthrop glacier to the right (north). Toward the volcano but in the intermediate foreground is the Steamboat Prow, which consists of coarse fragmental deposit. \\

\begin{figure}[h]
\centering
\includegraphics[width=0.8\textwidth]{080814_2.png}
\label{Fig:080814_2}
\caption{View of Rainier summit from Sunrise}
\end{figure}

\noindent 11:00 am -- Starting Hiking \\
We started hiking from Sunrise parking lot on to the Burroughs Mountain, which erupted sometime within the period 480,000-490,000 year ago. This is the earliest large volume lava flow from the onset of the modern Mt. Rainier edifice.

\subsection{Stop 3 -- Little ice age moraine of Emmons glacier, 11:30 am}
As we hiked up the mountain, we got a better view of the Emmons glacier valley. To the north side of the valley (right side in the picture) we saw a sharp scarp along the valley. Outside the scarp there are developed forest, but inside the scarp there are no trees on the slope and very small trees in the valley. It turns out to be the limit of the little ice age. During the peak of little ice age the valley up to the tree line was covered by ice sheet.  Coring of trees date the cessation of little ice age to be approximately 1850. 
Mike also pointed out the terminus of the Emmons glacier (Figure~\ref{Fig:080814_3}). It is at the base of a dirty, seemingly rocky slope where a stream flows out. 

\begin{figure}[h]
\centering
\includegraphics[width=0.8\textwidth]{080814_3.png}
\label{Fig:080814_3}
\caption{Emmons glacier and little ice age tree line.}
\end{figure}

\subsection{Stop 4 -- On top of 2nd Burroughs Mountain (elevation 7400 ft), 12:30 pm}
After we hiked to the top of the 2nd Burroghs Mtn, we took a closer look at Steamboat Prow. Steamboat Prow consists of coarse fragmented deposit that stratigraphically underlie the Burrows Mtn lava flow. The fragmental deposit dip away from the present summit as much as 20 degrees, which suggests that a high edifice existed early in the growth of modern Rainier. This means that modern Rainier either was built on a tall ancestral Rainier or grew very fast. \\\\
Paleomagnetic measurements on the deposited clasts show that those near the top of Steamboat Prow have uniform magnetic orientations (that they are deposited hot), and those at lower elevation have random magnetic orientations (that they are deposited cold). The explanation is that the initially hot pyroclastic flow entrained glacial deposit on the way down, which lowered its temperature and transformed it into a lahar
\\\\
1:00 -- 1:20 pm -- 	Lunch. 

\subsection{Stop 5 -- Pumice of the C tephra, 1:30 pm}

After deciding that we were not going to climb up to the 3rd Burroghs Mtn, we began descending down to the Glacial Basin Trail.  Near the beginning of the descend we stopped to look at the pumice from the C tephra deposit. This is from a 2200 ybp eruption, the most voluminous one in the Holocene (~0.1 cu km). This is a compositionally diverse eruption. Most of the deposit is brown-ish andesite. But there are light color dacite streaks in them, a possible result of magma mingling (Figure~\ref{Fig:080814_5}). Also there are grey-ish vesicular andesite. Geochemical analysis showed that both are very high in Sr, similar to the Sr concentration of a 2500 ybp eruption.  The interpretation is that these dacite streaks and vesicular andesite are residuals left over in the shallow magmatic system after the 2500 ybp eruption, and they were mixed with the 2200 ybp magma and flushed out of the system.

\begin{figure}[h]
\centering
\includegraphics[width=0.8\textwidth]{080814_5.png}
\label{Fig:080814_5}
\caption{Vesicular andesite (left) and regular andesite with white dacite blebs (right) at C tephra}
\end{figure}

\subsection{Stop 6 -- Igneous inclusion in andesite, 1:50 pm}
As we continue down the trail, near the base of the lava flow, we saw again some fine grain andesite enclaves that are reportedly geochemically more evolved than the surrounding andesite (Figure~\ref{Fig:080814_6}). They are interpreted to be result of magma mingling activity. 

\begin{figure}[h]
\centering
\includegraphics[width=0.8\textwidth]{080814_6.png}
\label{Fig:080814_6}
\caption{Plutonic inclusions in andesitic lava}
\end{figure}

\subsection{Stop 7 -- Back into granodiorite basement, 2:15 pm}
We continue hiking downhill. Shortly after the last stop (about 100 m elevation drop), we crossed a distinct boundary between the younger andesitic lava and older crystalline granodiorite basement. 
\\\\
3:00 -- We came down the Burroghs Mtn Trail to the junction with the Glacial Basin Trail. The team split into two groups. One continued on to Glacier Basin Trial(1.5 mi round trip), the other head back directly to the White River Campground parking lot.

\subsection{Stop 8 - Old mine at Glacier Basin}
We hiked up the Glacier Basin Trial. At the end of the trail we have a nice view of the Emmons Glacier (Figure~\ref{Fig:080814_8}). An old mine used to operate here, drilling into the basement granodiorite for useful minerals (e.g. sulphite). We also saw some vertical radial dikes in the lava flow.

\begin{figure}[h]
\centering
\includegraphics[width=0.8\textwidth]{080814_8.png}
\label{Fig:080814_8}
\caption{Glacial basin}
\end{figure}

\subsection{Stop 9 -- Back to White River Campground, 4:30 pm}
We returned to White River Campground, where we had left one car. Tom took this car along with three drivers back to Sunrise parking lot to retrieve the other cars. 
\\\\
5:30 pm -- Returned to campsite

\begin{figure}[h]
\centering
\includegraphics[width=0.8\textwidth]{080814_9.png}
\label{Fig:080814_9}
\caption{Trail map and approximate location of each stop}
\end{figure}

% Day 7 - August 9, 2014
\clearpage
\section{Day 7 -- August 9, 2014: }

% Day 8 - August 10, 2014
\clearpage
\section{Day 8 -- August 10, 2014: Mt. St. Helens}
\textbf{9:00 am:} Depart Iron Creek Campground, head towards Windy Ridge\\
\textbf{9:50 am:} Stop 1 -- Bear Meadow\\
\textbf{10:20  am:} Stop 2 -- Donnybrook Viewpoint\\
\textbf{10:35 am:}  Stop 3 -- Windy Ridge\\
\textbf{11:45 am:}  Stop 4 -- Begin hike along the Windy Ridge Trail\\
\textbf{12:50 pm:} Lunch at junction between Abraham and Windy Ridge Trails\\
\textbf{1:27 pm:} Stop 5 -- Arrive at Truman Trailhead and begin hike into the Pumice Plain\\
\textbf{3:55 pm:} Return to Windy Ridge parking lot and depart for Iron Creek\\
\textbf{4:15 pm:} Stop 6 -- Mine Car\\
\textbf{5:30 pm:}  Arrive at Iron Creek Campsite

\subsection{Stop 1 -- Bear Meadow, 9:50 am}

Here, Gary Rosenquist and his friends captured 22 pictures of the landslide and the following lateral blast at Mt. St. Helens on May 17, 1980 (Figure~\ref{Fig: 081014_1}). Activity prior to the eruption started as early as March 1980. As concern mounted scientists started to closely monitor the volcano. In these two months, the north side of the volcano rapidly expended, eventually reaching a rate of about 1 m/day just before the eruption. A spike in small magnitude earthquakes also was observed. Restricted areas were established around the volcano, however these were based on the assumption that the volcano would erupt vertically. The eruption started with an magnitude 5 earthquake which triggered a landslide. The mass movement of the landslide decompressed the magmatic system, which led to the following lateral blast. After the eruption, the group struggled to escape the area by driving east. The ash blocked the sunlight and rained on their windshield. It took them hours to drive out and luckily their engine was not choked by the ash.

\begin{figure}[htp]
	\centering
	\includegraphics[width=3in]{./stop1/1_1.jpg}
	\includegraphics[width=3in]{./stop1/1_2.jpg}
	\caption{Gary Rosenquist's photographs from the 1980 eruption, and the view of Mt. St. Helens from Bear Meadow.}
	\label{Fig: 081014_1}
\end{figure}

\subsection{Stop 2 -- Donnybrook Viewpoint, 10:30 am}

The 1980 eruption created hummocky deposits, which are observable from this viewpoint (Figure~\ref{Fig: 081014_2}). The eruption caused a lateral blast, which downed trees in the surrounding area so that they now lay in a radial pattern outwards from the volcano. Logs can still be seen floating in Spirit Lake over 30 years after the eruption. The blast brought material up to the ridge, which sadly killed the USGS geologist David Johnston, after he transmitted his famous last words: "Vancouver! Vancouver! This is it!''. His remains were never found. Johnston Ridge Observatory is named after him today.

\begin{figure}[htp]
	\centering
	\includegraphics[width=5in]{./stop2/2_1.jpg}
	\caption{Spirit Lake with tree deposits, valleys and ridges with hummocky terrian}
	\label{Fig: 081014_2}
\end{figure}

\subsection{Stop 3 -- Windy Ridge, 10:34 am}

Lecture by Heather Wright, USGS Volcano Observatory - Disaster Response and Mitigation Program (Figure~\ref{Fig: 081014_3_1}).
\\\\
Mt. St. Helens has been active since 300 ka; however, much of the edifice construction has been relatively young and most of the terrains we see today are less than 3000 years old. Several dome structures exist in the center of the  crater, including the 1100-1200 year old Sugar Bowl Dome and the dome that formed after the 1980 eruption (Figure~\ref{Fig: 081014_3_2}). 
\\\\
The eruption history of Mt. St. Helens can be divided into four different stages (from youngest to oldest): Spirit Lake, Swift Creek, Cougar, and Ape Cave. The most recent Spirit Lake stage is divided into six periods: Modern, Goat Rocks, Kalama, Sugar Bowl, Castle Creek, Pine Creek, and Smith Creek. The material from the Spirit Lake stage is more depleted in incompatible elements relative to other Cascade volcanos, suggesting that the source had already been depleted. During the Kalama Period the two largest eruptions contained more than 2 km$^3$ of erupted material, and the third largest had about 0.5 km$^3$. 
\\\\
Biologists have found that only two species of trees are still floating in Spirit Lake after the 1980 eruption, but most trees have sunk to the bottom of the lake where some are standing vertically. Due to the influx of decaying biologic material following the eruption, Spirit Lake became filled with bacteria such as E. coli, and even "human-brain eating'' bacteria. Fish were illegally restored to the lake, probably by a private airplane drop. 
\\\\
The most recent eruptive episode started in 2004, and several domes were produced from 2004-2008. These  domes pushed the crater glacier to make it the only advancing glacier in the North America. 
\\\\
At this location, Zach also gave us a short introduction to the iMUSH project, a seismic exploration of Mt. St. Helens.

\begin{figure}[htp]
	\centering
	\includegraphics[width=5in]{./stop3/3_1.jpg}
	\caption{USGS geologist Heather Wright and her students/postdocs. }
	\label{Fig: 081014_3_1}
\end{figure}

\begin{figure}[htp]
	\centering
	\includegraphics[width=5in]{./stop3/3_2.jpg}
	\caption{The view of Mt. St. Helens from this stop. The dome generated during the Sugar Bowl period can be seen in the center of the crater.}
	\label{Fig: 081014_3_2}
\end{figure}

\subsection{Stop 4 -- Windy Ridge Trail}
\subsubsection{Volcanic Breccia, 12:17 pm}

At this brief stop along the road from Windy Ridge we encountered volcanic breccia surrounded by pumice deposits (Figure~\ref{Fig: 081014_4}). It is likely not from the modern eruptive period, and may be a block and ash flow - a cooler debris flow than a pyroclastic flow.

\begin{figure}[htp]
	\centering
	\includegraphics[width=3in]{./stop4/4_1.jpg}
	\includegraphics[width=3in]{./stop4/4_2.jpg}
	\caption{Left: Old lava flow. Right: volcanic breccia.}
	\label{Fig: 081014_4}
\end{figure}

\subsubsection{Fall Deposit, 12:27 pm}

About 50m down the road we dug into some sedimentary layers (Figure~\ref{Fig: 081014_5}). The deposit layer we found here is well-sorted alternating fine and coarse grained material, indicating that it is air-transported, or fall deposit. Differences in grain size are likely due to the waxing and waning of the eruptive cycle.

\begin{figure}[htp]
	\centering
	\includegraphics[width=5in]{./stop5/5_1.jpg}
	\caption{Fall deposit from the volcano.}
	\label{Fig: 081014_5}
\end{figure}

\subsubsection{Truman Trailhead, 13:27}

The outcrop was comprised of low-density plagioclase-rich porphyritic rock (Figure~\ref{Fig: 081014_7}). These rocks are usually formed within the magmatic system over a long period of time as it slowly cools. The residence time of the magma can be determined from isotope analysis using Thorium-230 and Radium-226 dating. In the case of Mt. St. Helens, a crystallization time from 2000 yr - 1 Ma is detected. This can help to determine how much magma within the system is depleted and how much can stay molten during an eruption. 

The porphyritic texture of the rock is generated by water within the magma that forms bubbles as the pressure decreases near the surface. This loss of water also moves the solidus to higher temperatures, which causes the magma to crystalize the fine grained mineral microlite. If exhumation is too rapid, microlite crystals will not have time to form.

\begin{figure}[htp]
	\centering
	\includegraphics[width=5in]{./stop7/7_1.jpg}
	\caption{Low-density igneous rock with porphyritic texture.}
	\label{Fig: 081014_7}
\end{figure}

\subsection{Stop 5 -- Pumice Plain}

\subsubsection{Cryptodome Material, 1:51 pm}

This outcrop is within the blast zone of the 1980 eruption and was comprised mainly of fine-grained, crystalline, magmatic material (Figure~\ref{Fig: 081014_8_1}). We observed two kinds of igneous rocks with a lighter and a darker color, probably caused by different levels of water content (Figure~\ref{Fig: 081014_8_2}). The cooling of the magma body makes the allows the rocks to become brittle and break easily. This is part of the 1980 blast deposit.

\begin{figure}[htp]
	\centering
	\includegraphics[width=3in]{./stop8/8_1.jpg}
	\includegraphics[width=3in]{./stop8/8_2.jpg}
	\caption{Left: blast deposit, right: igneous rocks in the deposit.}
	\label{Fig: 081014_8_1}
\end{figure}

\begin{figure}[htp]
	\centering
	\includegraphics[width=3in]{./stop8/8_3.jpg}
	\includegraphics[width=3in]{./stop8/8_4.jpg}
	\caption{Left: cooling feature, right: light and dark bands.}
	\label{Fig: 081014_8_2}
\end{figure}

\subsubsection{Pyroclastic Flow Deposits, 2:07 pm}

The outcrop contains coarse to fine grained material and is very poorly sorted (Figure~\ref{Fig: 081014_9}). This indicates that it is a flow deposit.

\begin{figure}[htp]
	\centering
	\includegraphics[width=5in]{./stop9/9_1.jpg}
	\caption{Flow deposit}
	\label{Fig: 081014_9}
\end{figure}

\subsubsection{Kalama Period Deposits, 2:25 pm}

A lens of lighter material and some dead trees can be found within the outcrop at this location (Figure~\ref{Fig: 081014_10}). This outcrop contains deposits from the Kalama Period (1400-1500), but the lens is from the Goat Rocks period (1800) (Figure~\ref{Fig: 081014_11}). This deposit was ripped up and redeposited during the 1980 eruption, explaining why the rocks are not vertically in order.

\begin{figure}[htp]
	\centering
	\includegraphics[width=5in]{./stop10/10_1.jpg}
	\caption{Blast deposit}
	\label{Fig: 081014_10}
\end{figure}

\begin{figure}[htp]
	\centering
	\includegraphics[width=5in]{./stop11/11_1.jpg}
	\caption{deposition profile over 2000 years}
	\label{Fig: 081014_11}
\end{figure}

% Day 9 - August 11, 2014
\clearpage
\section{Day 9 -- August 11, 2014: }

% Day 10 - August 12, 2014
\clearpage
\section{Day 10 -- August 12, 2014: Mima Mounds and Ruby Beach}

\textbf{8:50 am:}  Leaving Millersylvania State Park campground
\\\textbf{9:15 am:}  Stop 1 -- Arriving at Mima Mounds Natural Area Preserve
\\\textbf{9:34 am:}  Start Mima Mounds Trail, ~ 0.7 mile
\\\textbf{10:00 am:}  Depart from Mima Mounds for Kalaloch Campground
\\\textbf{2:00 pm:}  Leave Kalaloch Campground for Ruby Beach
\\\textbf{2:15 pm:}  Stop 2 -- Arrive at Ruby Beach
\\\textbf{3:40 pm:}  Depart from Ruby Beach

\subsection{Stop 1 -- Mima Mounds Natural Area Preserve, 9:15 am}

The Mima Mounds Natural Area Preserve is a National Natural Landmark known for its mysterious mounded landscape, which is covered by prairie grassland.  The mounds are small regularly-spaced circular or elliptical hills about 10 meters apart and 2.5 meters in height (Figure~\ref{Fig:081214_1}). The mounds are composed primarily of gravel. There are about 8 to 10 mounds per acre, and the preserve covers 637 acres.\\\\
There are similar mounds in North America; pimple mounds by the Gulf of Mexico, prairie mounds in the northern Great Plains and hogwallow mounds in California (Figure~\ref{Fig:081214_2}).\\\\

\begin{figure}[h]
\centering
\includegraphics[width=0.8\textwidth]{081214_1.png}
\label{Fig:081214_1}
\caption{Mima Mounds}
\end{figure}

\begin{figure}
\centering
\makebox[\textwidth]{
\subfloat[]{\includegraphics[width=0.5\textwidth]{081214_2a.png}}
\subfloat[]{\includegraphics[width=0.5\textwidth]{081214_2b.png}}}
\caption{}
\label{Fig:081214_2}
\end{figure}

\noindent The origin of Mima Mounds has left scientists puzzled since the mid 19th century, and is now known for certain is that these mounds were formed after the glaciers of the Last Glacial Maximum started to recede about 16,500 years ago (see illustration above, right). Around 16,500 years ago, these glaciers moved into Washington, where their southernmost edge met the Mima Prairie. After the ice melted, the huge loads of rocks previously held were deposited across the area and became the prairie that we see today.
Several theories for the origin of Mima mounds have surfaced yet none has been proven conclusively. Of the more than 30 explanations, the most popular ones include:
\\1)	Glacial Ice
\\Gravels and stones washed onto glaciers were gathered in suncups/pits, and as the ice melted, they were left behind in mounds.
\\2)	Seismic activity 
\\Vibrational shock waves travel through soil, and the mounds peak where the wave peaks intersect, and mounds dip as wave cancel each other out. However, many earthquakes have taken place since, and none left similar mounds. 
\\3)	Fluvial deposition
\\The mounds are deposits from glacially dammed lake with sediment rich floods, and water flowed around vegetation where sediments collected.
\\4)	Erosion
\\Glacial meltwater eroded the soil between plants (trees and thrubs), leaving mounded sediments around the vegetation.
\\5)	Pocket gophers/ Fossorial Rodent Hypothesis
\\Small burrowing rodents, pocket gophers (Figure~\ref{Fig:081214_3}), built these mounds as shelters/nest chambers to stay above perched water tables. As the gophers excavated beneath the soil, they moved dense layers of glacial deposits upward, thus forming mounds. However, it has been argued that the mounds were there before the gophers moved into the area. 
\\6)	Permafrost cracking
\\Ice formed wedges in cracked permafrost (frozen ground), and as the ice melted, only mounded shaped soil remained.

\begin{figure}[h]
\centering
\includegraphics[width=0.5\textwidth]{081214_3.png}
\label{Fig:081214_3}
\caption{Pocket Gophers}
\end{figure}

\subsection{Stop 2 -- Ruby Beach, 2:15 pm}

Ruby beach is a pebble beach with sea stacks and tidal pools (Figure~\ref{Fig:081214_4}. On the beach, we observed a very large number of stranded velella, a genus of deep blue hydrozoans similar to the Portuguese man-o-war that are made up of a number of polyps. These live on the surface of the sea and are propelled by the action of the wind on a small sail-like protrusion that breaches the surface. Mass strandings of velellae occur every few years in this region, and are in general driven by changes in the direction of the wind that lead to these being washed ashore. There is a large amount of driftwood that collects on this beach thanks to the logging activity in the region. \\\\
In the northern section of the beach, there are cliffs as well as sea stacks. We found the sea stacks to be composed in some cases of fine-grained sandstone, and in other cases of a poorly-sorted conglomerate with angular clasts and large chunks of mudstone. These formations are olistostromes, or rocks that were consolidated after an underwater mudslide and subsequently exposed. \\\\
Another feature observed at Ruby beach was the sorting of pebbles along the tide line. As the rounder pebbles are more inclined to roll down the slope and into the water, the pebbles remaining on the beach further from the tide line are flatter or more irregularly shaped than those closer to the water.\\\\
At 3:40 PM, we left Ruby Beach for the campground, and had a short discussion to summarize the outcrops that we had seen in the context of the active plate margin of the area.

\begin{figure}[h]
\centering
\includegraphics[width=0.8\textwidth]{081214_4.png}
\label{Fig:081214_4}
\caption{Ruby Beach}
\end{figure}

% Day 11 - August 13, 2014
\clearpage
\section{Day 11 -- August 13, 2014: Olympic National Park and the Hoh Rainforest}

\textbf{9:05 am:}		Morning Discussion at Kalaloch Campground
\\\textbf{10:00 am:}		Stop 1 -- Forks Timber Museum
\\\textbf{12:10 pm:}		Stop 2 -- Hoh Rainforest � Hall of Mosses Trail (0.8 mi.)
\\\textbf{1:30 pm:}		Stop 3 -- Hoh Rainforest � Spruce Nature Trail (1.2 mi.)
\\\textbf{5:00 pm:}		Stop 4 -- Kalaloch Beach

\subsection{Stop 1 -- Forks Timber Museum, 10:00 am}

\subsection{Stop 2 -- Hall of Mosses Trail, Hoh Rainforest, 12:10 pm}
The hikes we did at the park were not strenuous and had magnificent views (Figure~\ref{Fig:081314_1}. We started with the Hall of Moss Trial, a 0.8 mile loop through old growth temperate rain forest, with 100 foot elevation gain. This trail exhibits more than 100 types of mosses, and there are guided tours open to the public, given by the forest rangers. The climate is moist when not raining and it is advised to bring a rain jacket. The region receives some of the highest precipitation in the USA thanks to the orographic effect of the Olympic mountains and Cascades blocking the moisture-laden westerly winds from the Pacific, and experiences up to 200 inches of rainfall per year.\\\\

\begin{figure}[h]
\centering
\includegraphics[width=0.8\textwidth]{081314_1.png}
\label{Fig:081314_1}
\caption{Map of Hoh Rainforest trails}
\end{figure}

\noindent Due to the high rates of precipitation, the moisture in the ground is retained near the surface, rather than at depth. This leads the trees of the rainforest to develop very shallow root systems in order to take advantage of this moisture. As a result, the trees, not being deep-rooted, are easily toppled by the wind during the frequent storms occurring in the region. It is estimated that approximately 80\% of tree deaths in this forest are caused by the wind. \\\\
The epiphytes found in this forest � primarily mosses and lichens � are a crucial part of the ecosystem (Figure~\ref{Fig:081314_2}). The relationship between the trees and epiphytes is symbiotic: the epiphytes use the trees for support and nutrients, while the trees have adapted to grow specialized lateral roots that re-absorb nutrients and moisture trapped in the moss and lichen layers. 

\begin{figure}
\centering
\makebox[\textwidth]{
\subfloat[]{\includegraphics[width=0.3\textwidth]{081314_2.png}}
\subfloat[]{\includegraphics[width=0.3\textwidth]{081314_3.png}}
\subfloat[]{\includegraphics[width=0.3\textwidth]{081314_4.png}}}
\caption{Mosses in the Hoh Rainforest}
\label{Fig:081314_2}
\end{figure}

\subsection{Stop 3 -- Spruce Nature Trail, Hoh Rainforest, 1:30 pm}
The second hike, Spruce Nature Trail, is a 1.2 mile loop through the forest and the Hoh River. It is fairly flat with less than 100 foot elevation gain.  The dominant tree species were the Sitka Spruce, Western Hemlock, and Douglas Fir; but other conifers and several deciduous species were present as well.  According to the forest rangers �Many are 100s years old and can reach 250 feet in height and 30 to 60 feet in circumference.� When the group tried to measure the circumference, it took 8 students to hug one tree! \\\\
This trail crosses both the glacier-fed Hoh river and the spring-fed Taft river, which are the primary waterbodies that irrigate the rainforest. The Hoh river, being glacier-fed, has a much more variable path and flow volume than the Taft river. \\\\
Signboards along this trail provide descriptions of the common tree species, and also point out features in the forest. Among these are clearings used as grazing pastures by elk, toppled trees, and nurse log colonnades. �Nurse logs� are the trunks of trees toppled by the wind. In an ecosystem where access to sunlight is highly competitive, these toppled trees provide a gap in the canopy that allow young saplings to grow rapidly as they are able to access sunlight. These saplings grow along the trunk of the fallen tree, using up the nutrients present in the organic matter. This results in sets of trees of approximately the same age growing in near-perfect rows known as nurse log colonnades (Figure~\ref{Fig:081314_3}). In some cases, the remains of the fallen tree are still visible. It is estimated that 98\% of the trees in the Hoh rainforest began as saplings on nurse logs. 

\begin{figure}[h]
\centering
\includegraphics[width=0.8\textwidth]{081314_5.png}
\label{Fig:081314_3}
\caption{Row of trees grown on a nurse log}
\end{figure}

\subsection{Stop 4 -- Kalaloch Beach outcrop, 5:00 pm}
The group speculated on its sequence of formation. We observed several layers of till that were formed as intermittent growth and shrinkage of glaciers in the Olympics, corresponding with the last interglacial period through the glacial sequence to present. This was deduced after pieces of wood within the till, and geochemical measurements were analyzed. When stepping back, to put things in perspective, a large thickness variation that was caused by the active plate margin setting was evident. The structure was broadly folded in the North-South direction, correlating with the compensation of the rotation of the plates. It is folded North-South.


% Day 12 - August 14, 2014
\clearpage
\section{Day 12 - August 14, 2014}
We drove to the airport.

\clearpage
\bibliographystyle{plainnat}
\bibliography{citations_fieldtrip}


\end{document}